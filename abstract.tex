\begin{flushleft}

\textbf{\textsc{\LARGE General Abstract}}

\vspace{0.2 in}

As fireballs, more commonly known as shooting stars, fly through Earth’s atmosphere at breakneck speeds, they emit light that can be observed from our planet’s surface. 
Willamette’s D6 AllSky Survey is a camera system that probes the night sky for these events.
One method of quantifying the number of events seen is through what is called the flux.
Flux describes the number of events that occur per unit time per unit area.
Because one camera system can only observe $<0.4\%$ of Earth’s total sky area, amateur astronomers hold a significant role in fireball observations.
Their observations provide a robust data sample size that can then be used to gain a deeper understanding of the flux rates and property distributions of fireballs.
While complex multi-camera professional systems currently exist, there is need for more economic, accessible, and versatile systems. 
We will discuss the feasibility of our current observational setup and how it compares to more elaborate existing systems.
\vspace{0.25 in}

\textbf{\textsc{\LARGE Technical Abstract}}

\vspace{0.2 in}

Fireballs, more technically known as bolides, are recognizable by the light they emit through ablation in the atmosphere.. 
The D6 AllSky Camera was designed by Dr.~Jed Rembold and Kyle McSwain as an alternative observation system for fireball research. 
It is smaller, cheaper, and much more portable, than most existing systems used by professional astronomers. 
By measuring and comparing distributions and the average flux of fireballs to other observation networks, we aim to assess the feasibility and efficiency of using the D6 AllSky Camera for fireball research.  
Due to poor weather and other complications, our observed data sample this year was not sufficiently large to produce a reliable average flux rate.
However the versatile framework and software pipeline established in this research will assuredly aid in the analysis of fireballs throughout further observations with the D6 AllSky Camera.

\end{flushleft}

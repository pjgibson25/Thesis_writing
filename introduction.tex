\chapter{Introduction}

Based on a rough estimate, there are about \num{10000} trillion ants, \num{7.6} billion humans, and a few million elephants in the world.
When considering this small\comment{these numbers are small?!} data set, one might come to the conclusion that there are more small objects in the world than there are large objects.
It so happens that this conclusion not only holds true on Earth, but it also holds true in our universe.\comment{I think you need a sentence to bridge these two.}
In our solar system, there is one star, eight planets, and an almost incomprehensible number of small rocks traveling through space. 

Similarly to distances in space, velocities of objects in space are of higher orders of magnitude than those observed on earth.\comment{So I get what you are going for here, but this is a convoluted sentence.}
For example, the Earth travels at approximately \SI{30}{\kilo\meter\per\second} while small rocks can travel between \SIrange{11}{70}{\kilo\meter\per\second}.  
The speeds of these rocks exceed the muzzle velocity of a bullet.\comment{probably want a citation here}
When we observe a meteor shower, we are witnessing a barrage of these bullet-like rocks.  
Fortunately for mankind, Earth’s atmosphere provides a protective shield consisting of a condensed array of particles.\comment{The start of this sentence is good but I don't like the end as much}
Condensed is a term being used in relativity to the vacuum of space in which these rocks spend the majority of their lifetime.\comment{I'd just bake this definition into the previous sentence}

As a rocky object travels through Earth’s atmosphere, it collides with particles and burns in a phenomena known as ablation.  
The result is a release of energy in the form of both heat and light.  
The objects, which ignite in a fiery ball are called fireballs.\comment{Careful here, technically fireballs are only the brightest meteors} 
For large enough fireballs, the light produced in ablation can be seen from the human eye as shooting stars.\comment{I'd remove this sentence and provide more context about the different magnitudes they might be seen at.}

By observing the photometric (visual) magnitude, duration, and other properties of individual fireball events, an observer can estimate impact energies and sizes of these near-Earth rocky objects.
Given a large enough sample size, observers can also determine the flux, or the number of events within a specific area per time, of fireballs of varying sizes and energies.\comment{Paragraph only has two sentences.}

While cameras set up by organizations such as NASA and the Spanish Meteor Network (SPMN) provide useful and precise measurements of fireball events, they lack flexibility and affordability.  
Often rendered immobile due to their connection to powerful computers, these systems provide only a small piece to the puzzle of fireball observation.\comment{What do you mean by small piece here?}
Any individual camera system can only observe up to around 0.03 percent of earth’s total sky.\comment{Oh, is this what you meant? I feel like this paragraph needs something more then.} 

This project aims to analyze the feasibility of the Willamette D6 AllSky camera, a new alternative camera system for conducting fireball research. 
Occupying about the same space as a traffic cone, this camera is easily transportable and can be replicated at a fraction of the price of the more expensive professionally used systems.\comment{Should be careful with language here, the camera itself is basically the same as in the other systems. The supporting infrastructure is what differs.}
By comparing flux rates measured through the analysis of data taken from Willamette’s D6 AllSky camera to more well-recognized systems, we will determine if our setup is a practical option for amateur astronomers interested in contributing to fireball research.\comment{Possibly break this sentence up and clean up the wording. Its possibly the most important sentence in your introduction, so don't skimp on it.}

This paper is broken up into several sections for ease of reading. Chapter 2 details useful information surrounding fireballs, their importance, existing research, and the theory necessary to calculate flux rates. \cite{trigo-rodriguez_2006_2007}\comment{Wait, why is this cited here? It seems super out of place...}
